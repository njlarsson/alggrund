% Created 2017-09-01 Fri 10:24
% Intended LaTeX compiler: lualatex
\documentclass[a4paper]{article}
\usepackage{graphicx}
\usepackage{grffile}
\usepackage{longtable}
\usepackage{wrapfig}
\usepackage{rotating}
\usepackage[normalem]{ulem}
\usepackage{amsmath}
\usepackage{textcomp}
\usepackage{amssymb}
\usepackage{capt-of}
\usepackage{hyperref}
\usepackage[swedish]{babel}
\usepackage{fontspec}
\setmainfont[Ligatures=TeX]{Linux Libertine O}
\usepackage{enumerate}
\frenchspacing
\setcounter{secnumdepth}{0}
\author{Jesper larsson \enspace \textit{v. 1.01 (aug. 2017)}}
\date{}
\title{Potenser och logaritmer}
\hypersetup{
 pdfauthor={Jesper larsson \enspace \textit{v. 1.01 (aug. 2017)}},
 pdftitle={Potenser och logaritmer},
 pdfkeywords={},
 pdfsubject={},
 pdfcreator={Emacs 25.1.1 (Org mode 9.0.9)}, 
 pdflang={Swedish}}
\begin{document}

\maketitle

\section*{Potens/exponentiering}
\label{sec:org541d7b1}

En \emph{potens} i matematiken skrivs \(x^{n}\), (\emph{\(x\) upphöjt till \(n\)}) där
\(x\) kallas \emph{basen} och \(n\) \emph{exponenten}. Sådan \emph{exponentiering} är en kortform
för upprepad multiplikation:

\[x^n = \underbrace{x \cdot x \cdot x\cdot \quad \cdots \quad \cdot x}_{n\text{ stycken }x}\]

Direkt av definitionen får vi följande. Kontrollera varför det blir så genom att skriva om potenserna som
upprepad multiplikation!
\[x^n \cdot x^m = x^{n+m}\]
och
\[\dfrac{x^n}{x^m} = x^{n-m}\]  

För bekvämlighets skull tillåter exponenten att vara mindre än ett, enligt
definition: \(x^0 = 1\) för alla \(x\) utom \(0\), och \(x^{-n} = 1 / x^n\). Kontrollera
att det stämmer med formlerna ovan!

Exponenten behöver inte alltid vara ett heltal. Vi kan till exempel hitta en
vettig betydelse för \(x^{1/2}\). Vi bör ju ha \(x^{1/2}\cdot x^{1/2} = x^{1/2 +
1/2} = x\). Det uppfylls bara om vi definierar \(x^{1/2} = \sqrt{x}\). Enligt samma
argument är \(x^{1/3} = \sqrt[3]{x}\).

\section*{Logaritm}
\label{sec:org8da2d70}

\emph{Logaritmering} är motsatsen till exponentiering. Vi definierar logaritmen i bas
\(b\), eller 
\(b\text{\emph{-logaritmen}}\), av \(x\) som det tal \(n\) som gör att \(b^{n}=x\). Alltså:
\[\log_b x = n \qquad \text{betyder att} \qquad b^n = x\]

Memorera den definitionen! Kan också uttryckas \(b^{log_b x} = x\) (tänk
efter varför).

Ett mer handfast sätt att förklara logaritm är följande: om
man börjar med \(x\), delar med \(b\), sedan delar resultatet av det med \(b\), och
fortsätter att dela med \(b\) tills man får \(1\), 
så är \(\log_b x\) antalet divisioner man har utfört. Det funkar
dock bara om \(x\) är en potens av \(b\), alltså \(x=b^n\) för något heltal \(n\), annars
kommer man aldrig exakt till \(1\).  (Tänk efter varför den här förklaringen
stämmer med definitionen.)

\subsection*{Tvålogaritm}
\label{sec:orgaa845a5}

   De flesta har antagligen stött på \emph{tiologaritmen} (intressant för att tio är
basen i vårt vanliga talsystem) och den \emph{naturliga logaritmen}, med
basen \(e \approx 2.71828\) (intressant därför att \(x^e\), \emph{exponentialfunktionen},
har speciella matematiska egenskaper). Men när vi datavetare talar om logaritm
menar vi för det mesta \emph{tvålogaritmen}. Därför definierar vi en kortform för \(2\text{-logaritmen}\):
\[\lg x = \log_2 x\]
Det är inte alla som använder \(\lg\) som \(\log_2\), men vår kursbok och många
andra gör det.

\subsection*{Avrundning}
\label{sec:org57038d2}

När \(x\) är en potens av två (alltså \(x = 2^n\) för något heltal \(n\)),
är \(\lg x\) ett heltal. Men det är meningsfullt att tala om \(\lg x\) för andra tal
också. Bara som ett exempel vet vi ju redan att \(2^{1/2} =
\sqrt{2}\), så \(\lg (\sqrt{2})
= 1/2\).

Den generella definitionen av logaritmer för positiva tal hoppar vi över,
men vi kan behöva resonera om \emph{mellan} vilka värden de måste ligga. Om a, b och
c är tre positiva tal sådana att \(a < b < c\) så gäller också \(\lg a < \lg b
< \lg c\) (tänk efter varför). Är vi till exempel intresserade av värdet på \(\lg 10\) är kan vi
konstatera att \(2^3 < 10 < 2^4\), och att \(\lg 10\) därför måste vara ett tal
mellan \(3\) och \(4\). Ofta kan vi nöja oss så, och använda den här
notationen för att avrunda upp eller ner:
\begin{align*}
\lfloor x \rfloor: & \quad \text{\emph{golvet} av $x$: det största heltalet $ \le x$ \enskip (avrundning nedåt)} \\
\lceil x \rceil: & \quad \text{\emph{taket} av $x$: det minsta heltalet $ \ge x$ \enskip (avrundning uppåt)}
\end{align*}
Så exempelvis är \(\lfloor \lg 10 \rfloor = 3\) och \(\lceil \lg 10 \rceil = 4\).

\subsection*{Viktiga logaritmlagar}
\label{sec:org2959031}

Följande formler följer av definitionen av logaritm (här anges ingen bas för logaritmerna eftersom formlerna gäller oavsett vilken
logaritm man menar):
\begin{eqnarray*}
  \log (x\cdot y) &=& \log x + \log y \\
  \log x^y &=& y\log x
\end{eqnarray*}
Försök att själv klura ut varför det blir så, och memorera sedan dessa
lagar. De är mycket användbara att kunna utantill för att snabbt kunna använda.

\subsection*{Basbyte}
\label{sec:org08d5dc3}

En sista viktig formel: eftersom \(x = a^{\log_a x}\):
\[\log_b x = \log_b a^{\log_a x} = (\log_b a)\cdot (\log_a x)\]
Det betyder, eftersom \(\log_b a\) är en konstant som bara beror på \(a\) och \(b\),
att logaritmer i olika baser bara skiljer sig åt med en konstant faktor, alltså
\[\log_b x = k\cdot \log_a x\]
där \(k\) är en konstant som bestäms av \(a\) och \(b\).

Exempelvis är \(\lg x \approx 3,32\cdot\log_{10}x\) (vilket innebär att ett tal i
det binära talsystemet får ungefär \(3,32\) gånger fler siffror än samma tal i det vanliga decimala).

I algoritmanalysjargong kan man uttrycka det så att alla logaritmer
\emph{asymptotiskt växer lika snabbt}. Därför kan man ibland nöja sig med att säga
\(\log\) utan att specificera basen.
\end{document}